\PassOptionsToPackage{unicode=true}{hyperref} % options for packages loaded elsewhere
\PassOptionsToPackage{hyphens}{url}
%
\documentclass[]{article}
\usepackage{lmodern}
\usepackage{amssymb,amsmath}
\usepackage{ifxetex,ifluatex}
\usepackage{fixltx2e} % provides \textsubscript
\ifnum 0\ifxetex 1\fi\ifluatex 1\fi=0 % if pdftex
  \usepackage[T1]{fontenc}
  \usepackage[utf8]{inputenc}
  \usepackage{textcomp} % provides euro and other symbols
\else % if luatex or xelatex
  \usepackage{unicode-math}
  \defaultfontfeatures{Ligatures=TeX,Scale=MatchLowercase}
\fi
% use upquote if available, for straight quotes in verbatim environments
\IfFileExists{upquote.sty}{\usepackage{upquote}}{}
% use microtype if available
\IfFileExists{microtype.sty}{%
\usepackage[]{microtype}
\UseMicrotypeSet[protrusion]{basicmath} % disable protrusion for tt fonts
}{}
\IfFileExists{parskip.sty}{%
\usepackage{parskip}
}{% else
\setlength{\parindent}{0pt}
\setlength{\parskip}{6pt plus 2pt minus 1pt}
}
\usepackage{hyperref}
\hypersetup{
            pdftitle={ Graphe - Devoir Maison propagation du Covid-19},
            pdfborder={0 0 0},
            breaklinks=true}
\urlstyle{same}  % don't use monospace font for urls
\setlength{\emergencystretch}{3em}  % prevent overfull lines
\providecommand{\tightlist}{%
  \setlength{\itemsep}{0pt}\setlength{\parskip}{0pt}}
\setcounter{secnumdepth}{0}
% Redefines (sub)paragraphs to behave more like sections
\ifx\paragraph\undefined\else
\let\oldparagraph\paragraph
\renewcommand{\paragraph}[1]{\oldparagraph{#1}\mbox{}}
\fi
\ifx\subparagraph\undefined\else
\let\oldsubparagraph\subparagraph
\renewcommand{\subparagraph}[1]{\oldsubparagraph{#1}\mbox{}}
\fi

% set default figure placement to htbp
\makeatletter
\def\fps@figure{htbp}
\makeatother


\title{Graphe - Devoir Maison propagation du Covid-19}
\providecommand{\subtitle}[1]{}
\subtitle{Decou Nathan , Delpeuch Sébastien, Moinel Aurélien, Pringalle Antoine}
\date{}

\begin{document}
\maketitle

Le but de ce devoir maison est d'utiliser l'algorithmique des graphes
pour modéliser une version simpliste de la propagation du Covid-19. Le
devoir se décompose en 3 parties d'implémentation, tout d'abord nous
mettons en place les différentes paramètres (mortalité, durée de
maladie) et les différents graphes. Vient ensuite le rajout de
paramètres simples comme des tests sur la population etc. Finalement
nous avons tenté de rajouter des facteurs plus réalistes à compléter.

Notre travail est résumé dans l'outil de visualisation que nous avons
crée disponible en fin de page.

\hypertarget{partie-i---impluxe9mentation-dune-base}{%
\subsection{\texorpdfstring{ Partie I - Implémentation d'une
base}{ Partie I - Implémentation d'une base}}\label{partie-i---impluxe9mentation-dune-base}}

La première partie permet de définir les bases de la modélisation c'est
à dire les différents états, les différents paramètres, les règles de
changement d'état, les topologies de graphes étudiés et les différents
modèles de ces derniers. La population est pour l'instant considérée
comme uniforme (chaque individu est sujet à la maladie de la même
manière), ils peuvent être dans quatre états, Sain (\[S\]), Infecté
(\[M\]), Guéri et donc imunisés (\[G\]) ou décédé (\[D\]). De plus
chaque personne suit les règles suivantes + Si une personne saine \[X\]
fréquente une personne malade \[Y\] alors, avec la probabilité \[q\],
\[X\] devient malade par l'intermédiaire de \[Y\] pour une durée \[r\].
+ Si une personne est malade depuis \[r\] jours alors soit elle décède
avec une probabilité \[p\] soit elle devient immunisé avec une
probabilité \[(1-p)\]. + Une personne immunisée ou décédée ne change
jamais d'état. + Une personne décédée ne peut pas contaminer une
personne saine.

Nous allons aussi définir les différents graphes de contact permettant
de simuler les différentes relations de chaque individu, trois graphes
sont alors mis en oeuvre + Un graphe circulaire, les personnes sont
numérotés de 1 à \[n\] et chaque personne, numérotés \[i\], est relié
aux personnes \[(i-1)\] et \[(i+1)\] modulo \[n\],

\hypertarget{partie-ii---rajout-de-tests-simples}{%
\subsection{\texorpdfstring{ Partie II - Rajout de tests
simples}{ Partie II - Rajout de tests simples}}\label{partie-ii---rajout-de-tests-simples}}

\hypertarget{partie-iii---rajouts-de-facteurs-plus-ruxe9alistes}{%
\subsection{\texorpdfstring{ Partie III - Rajouts de facteurs plus
réalistes}{ Partie III - Rajouts de facteurs plus réalistes}}\label{partie-iii---rajouts-de-facteurs-plus-ruxe9alistes}}

\hypertarget{partie-iv---outil-de-visualisation}{%
\subsection{\texorpdfstring{ Partie IV - Outil de
visualisation}{ Partie IV - Outil de visualisation}}\label{partie-iv---outil-de-visualisation}}

Exercice avec Python

Page Web interactive

Cette page est associée à un script Python

\end{document}
